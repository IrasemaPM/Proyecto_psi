% Options for packages loaded elsewhere
\PassOptionsToPackage{unicode}{hyperref}
\PassOptionsToPackage{hyphens}{url}
\PassOptionsToPackage{dvipsnames,svgnames,x11names}{xcolor}
%
\documentclass[
  letterpaper,
  DIV=11,
  numbers=noendperiod]{scrreprt}

\usepackage{amsmath,amssymb}
\usepackage{iftex}
\ifPDFTeX
  \usepackage[T1]{fontenc}
  \usepackage[utf8]{inputenc}
  \usepackage{textcomp} % provide euro and other symbols
\else % if luatex or xetex
  \usepackage{unicode-math}
  \defaultfontfeatures{Scale=MatchLowercase}
  \defaultfontfeatures[\rmfamily]{Ligatures=TeX,Scale=1}
\fi
\usepackage{lmodern}
\ifPDFTeX\else  
    % xetex/luatex font selection
\fi
% Use upquote if available, for straight quotes in verbatim environments
\IfFileExists{upquote.sty}{\usepackage{upquote}}{}
\IfFileExists{microtype.sty}{% use microtype if available
  \usepackage[]{microtype}
  \UseMicrotypeSet[protrusion]{basicmath} % disable protrusion for tt fonts
}{}
\makeatletter
\@ifundefined{KOMAClassName}{% if non-KOMA class
  \IfFileExists{parskip.sty}{%
    \usepackage{parskip}
  }{% else
    \setlength{\parindent}{0pt}
    \setlength{\parskip}{6pt plus 2pt minus 1pt}}
}{% if KOMA class
  \KOMAoptions{parskip=half}}
\makeatother
\usepackage{xcolor}
\setlength{\emergencystretch}{3em} % prevent overfull lines
\setcounter{secnumdepth}{5}
% Make \paragraph and \subparagraph free-standing
\makeatletter
\ifx\paragraph\undefined\else
  \let\oldparagraph\paragraph
  \renewcommand{\paragraph}{
    \@ifstar
      \xxxParagraphStar
      \xxxParagraphNoStar
  }
  \newcommand{\xxxParagraphStar}[1]{\oldparagraph*{#1}\mbox{}}
  \newcommand{\xxxParagraphNoStar}[1]{\oldparagraph{#1}\mbox{}}
\fi
\ifx\subparagraph\undefined\else
  \let\oldsubparagraph\subparagraph
  \renewcommand{\subparagraph}{
    \@ifstar
      \xxxSubParagraphStar
      \xxxSubParagraphNoStar
  }
  \newcommand{\xxxSubParagraphStar}[1]{\oldsubparagraph*{#1}\mbox{}}
  \newcommand{\xxxSubParagraphNoStar}[1]{\oldsubparagraph{#1}\mbox{}}
\fi
\makeatother


\providecommand{\tightlist}{%
  \setlength{\itemsep}{0pt}\setlength{\parskip}{0pt}}\usepackage{longtable,booktabs,array}
\usepackage{calc} % for calculating minipage widths
% Correct order of tables after \paragraph or \subparagraph
\usepackage{etoolbox}
\makeatletter
\patchcmd\longtable{\par}{\if@noskipsec\mbox{}\fi\par}{}{}
\makeatother
% Allow footnotes in longtable head/foot
\IfFileExists{footnotehyper.sty}{\usepackage{footnotehyper}}{\usepackage{footnote}}
\makesavenoteenv{longtable}
\usepackage{graphicx}
\makeatletter
\def\maxwidth{\ifdim\Gin@nat@width>\linewidth\linewidth\else\Gin@nat@width\fi}
\def\maxheight{\ifdim\Gin@nat@height>\textheight\textheight\else\Gin@nat@height\fi}
\makeatother
% Scale images if necessary, so that they will not overflow the page
% margins by default, and it is still possible to overwrite the defaults
% using explicit options in \includegraphics[width, height, ...]{}
\setkeys{Gin}{width=\maxwidth,height=\maxheight,keepaspectratio}
% Set default figure placement to htbp
\makeatletter
\def\fps@figure{htbp}
\makeatother
% definitions for citeproc citations
\NewDocumentCommand\citeproctext{}{}
\NewDocumentCommand\citeproc{mm}{%
  \begingroup\def\citeproctext{#2}\cite{#1}\endgroup}
\makeatletter
 % allow citations to break across lines
 \let\@cite@ofmt\@firstofone
 % avoid brackets around text for \cite:
 \def\@biblabel#1{}
 \def\@cite#1#2{{#1\if@tempswa , #2\fi}}
\makeatother
\newlength{\cslhangindent}
\setlength{\cslhangindent}{1.5em}
\newlength{\csllabelwidth}
\setlength{\csllabelwidth}{3em}
\newenvironment{CSLReferences}[2] % #1 hanging-indent, #2 entry-spacing
 {\begin{list}{}{%
  \setlength{\itemindent}{0pt}
  \setlength{\leftmargin}{0pt}
  \setlength{\parsep}{0pt}
  % turn on hanging indent if param 1 is 1
  \ifodd #1
   \setlength{\leftmargin}{\cslhangindent}
   \setlength{\itemindent}{-1\cslhangindent}
  \fi
  % set entry spacing
  \setlength{\itemsep}{#2\baselineskip}}}
 {\end{list}}
\usepackage{calc}
\newcommand{\CSLBlock}[1]{\hfill\break\parbox[t]{\linewidth}{\strut\ignorespaces#1\strut}}
\newcommand{\CSLLeftMargin}[1]{\parbox[t]{\csllabelwidth}{\strut#1\strut}}
\newcommand{\CSLRightInline}[1]{\parbox[t]{\linewidth - \csllabelwidth}{\strut#1\strut}}
\newcommand{\CSLIndent}[1]{\hspace{\cslhangindent}#1}

\KOMAoption{captions}{tableheading}
\makeatletter
\@ifpackageloaded{bookmark}{}{\usepackage{bookmark}}
\makeatother
\makeatletter
\@ifpackageloaded{caption}{}{\usepackage{caption}}
\AtBeginDocument{%
\ifdefined\contentsname
  \renewcommand*\contentsname{Table of contents}
\else
  \newcommand\contentsname{Table of contents}
\fi
\ifdefined\listfigurename
  \renewcommand*\listfigurename{List of Figures}
\else
  \newcommand\listfigurename{List of Figures}
\fi
\ifdefined\listtablename
  \renewcommand*\listtablename{List of Tables}
\else
  \newcommand\listtablename{List of Tables}
\fi
\ifdefined\figurename
  \renewcommand*\figurename{Figure}
\else
  \newcommand\figurename{Figure}
\fi
\ifdefined\tablename
  \renewcommand*\tablename{Table}
\else
  \newcommand\tablename{Table}
\fi
}
\@ifpackageloaded{float}{}{\usepackage{float}}
\floatstyle{ruled}
\@ifundefined{c@chapter}{\newfloat{codelisting}{h}{lop}}{\newfloat{codelisting}{h}{lop}[chapter]}
\floatname{codelisting}{Listing}
\newcommand*\listoflistings{\listof{codelisting}{List of Listings}}
\makeatother
\makeatletter
\makeatother
\makeatletter
\@ifpackageloaded{caption}{}{\usepackage{caption}}
\@ifpackageloaded{subcaption}{}{\usepackage{subcaption}}
\makeatother

\ifLuaTeX
  \usepackage{selnolig}  % disable illegal ligatures
\fi
\usepackage{bookmark}

\IfFileExists{xurl.sty}{\usepackage{xurl}}{} % add URL line breaks if available
\urlstyle{same} % disable monospaced font for URLs
\hypersetup{
  pdftitle={El valor optimo del indice de comportamiento},
  pdfauthor={Irasema Pedroza Meza},
  colorlinks=true,
  linkcolor={blue},
  filecolor={Maroon},
  citecolor={Blue},
  urlcolor={Blue},
  pdfcreator={LaTeX via pandoc}}


\title{El valor optimo del indice de comportamiento}
\author{Irasema Pedroza Meza}
\date{2024-08-12}

\begin{document}
\maketitle

\renewcommand*\contentsname{Table of contents}
{
\hypersetup{linkcolor=}
\setcounter{tocdepth}{2}
\tableofcontents
}

\bookmarksetup{startatroot}

\chapter*{Prefacio}\label{prefacio}
\addcontentsline{toc}{chapter}{Prefacio}

\markboth{Prefacio}{Prefacio}

En este libro se desarrollara el proyecto de la clase de Aprendizaje
Reforzado.

\bookmarksetup{startatroot}

\chapter{Introdución}\label{introduciuxf3n}

En la literatura existen una variedad de modelos en los cuales
consideran la vacunación como medida de prevención para el control de
enfermedades {[}Jorge, y mas modelos{]}. Sin embargo, este medio de
control llega a presentar algunas fallas como son: la falla de grado,
falla en la toma y falla en la duración {[}Maclean{]}. En la modelación
de enfermedades respiratorias con ecuaciones diferenciales, se suele
considerar que las vacunas tienen una falla tipo de grado {[}Jorge y mas
modelos{]}. Por otro lado, Pedroza, et. el, proponen un modelo de
ecuaciones diferenciales donde consideran que la vacuna tiene dos tipos
de fallas: la falla de grado y la falla en la toma.

En este ultimo trabajo, proponen un índice de comportamiento (\(\psi\))
el cual permite medir que tan riguroso pueden seguir las medidas de
prevención una vez que son vacunados. En el modelo que proponen este
índice solo afecta a los vacunados no inmunes

\bookmarksetup{startatroot}

\chapter{Formulación del Proceso de Decisión de
Markov}\label{formulaciuxf3n-del-proceso-de-decisiuxf3n-de-markov}

Los estados de nuestro proceso de decisión de Markov representarán la
proporción de la población en cada categoría del modelo propuesto por
XXX:

\begin{itemize}
\item
  \(S_{t}\): Fracción de susceptibles en el tiempo \(t\).
\item
  \(V_{+t}\): Fracción de vacunados inmunes en el tiempo \(t\).
\item
  \(V_{-t}\): Fracción de vacunados no inmunes en el tiempo \(t\).
\item
  \(I_{t}\): Fracción de infectados en el tiempo \(t\).
\item
  \(R_{t}\): Fracción de recuperados en el tiempo \(t\).
\end{itemize}

El estado global del sistema en el tiempo \(t\) se presenta como

\[
x_{t} = (S_{t}, V_{+t}, V_{-t}, I_{t}, R_{t})
\]

El escenario que consideraremos para cada \(t \in {0,1,\dots, N}\)para
el proceso de Markov:

\begin{itemize}
\tightlist
\item
  \(x_{t}\): representa la dinámica de la enfermedad en el tiempo \(t\).
\item
  \(a_{t}\): representa en qué escenario del índice de comportamiento se
  encuentra la población en el tiempo\(t\).
\end{itemize}

Algunos supuestos que estaremos considerando para nuestro proceso son:

\begin{itemize}
\tightlist
\item
  Las personas cambian su comportamiento en el tiempo \(t\) de forma
  instantánea.
\item
  Las únicas personas que pueden cambiar su comportamiento son los
  vacunados no inmunes.
\item
  Supondremos que las personas cambian su comportamiento bajo una
  distribución uniforme \([0.5, 2]\).
\end{itemize}

Bajo los supuestos anteriormente mencionados, consideramos el siguiente
Modelo de Control de Markov.

\[
(\mathbf{X},  \{A(x): x \in X\}, \mathbf{P}, \mathbf{C})
\]

donde \(\mathbf{X}\) es el espacio de los estados, \$ \{A(x): x
\in X\}\$ es el espacio de las acciones admisibles, \(\mathbf{P}\) es la
ley de transicion de modelo y \(\mathbf{C}\) es la funcion de costo.

\bookmarksetup{startatroot}

\chapter{Dinámica del Modelo}\label{dinuxe1mica-del-modelo}

En esta sección veremos cómo evoluciona el sistema atreves del tiempo
\(t\). La dinámica de la enfermedad esta representando por la siguiente
cadena de Markov

\begin{itemize}
\item
  \(S_{t+1}= \mu -\beta S_{t}I_{t}- (\mu + \phi)S_{t}+\omega V_{+t}+\xi R_{t}\)
\item
  \(V_{+t+1}= \phi\_{+}(\sigma)S_{t}-(\mu+\omega) V_{+t}\)
\item
  \(V_{-t+1}=\phi_{-}(\sigma)S_{t}- \psi(1-\sigma) \beta V_{-t} I_{t}-\mu V_{-t}\)
\item
  \(I_{t+1}= \beta S_{t}I_{t}+\psi(1-\sigma) \beta V_{-t} I_{t} -(\mu+ \gamma)I_{t}\)
\item
  \(R_{t+1}=\gamma I_{t}-(\mu +\xi) R_{t}\)
\end{itemize}

En la (\textbf{diagramamodel?}) se observa un

\phantomsection\label{diagramamodel}\href{EstadosSVVIRS.JPG}{Diagrama}

Las acciones representan los valores del índice de comportamiento
\((\psi)\). Supondremos que los valores del índice de comportamiento
están restringidos al intervalo \([0.5,2]\). Las acciones consideradas
en nuestro modelo están dividas en tres principales acciones:

\begin{itemize}
\item
  Si \(\psi\in [0.5,0.9)\) entonces diremos que las personas se portan
  bien, es decir, que las personas siguen las medidas de prevención de
  forma estricta.
\item
  Si \(\psi\in [0.9,1.2]\) entonces diremos que las personas se portan
  normal, es decir, que las personas siguen las medidas de prevención.
\item
  Si \(\psi\in (1.2,2]\) entonces diremos que las personas se portan
  mal, es decir, que las personas no siguen las medidas de prevención.
\end{itemize}

Supondremos que las personas toman decisiones diarias mediante unas
distribución \(x_{t} Uni[0.5,2]\) el periodo de observación será de un
año. Como se ha mencionado con anterioridad el objetivo de este trabajo
es encontrar el valor de \(\psi\) que permita tener la incidencia
acumulada más pequeña al final del año. Entonces la probabilidad de
tener \(i\) de incidencia acumulada y cambiar el comportamiento
\(\alpha\) en el tiempo \(t\) y pasar al estado \(j\) está dada por:
\[P_{ij}(\alpha)=P[x_{t+1}=j|x_{t}=i,a_{t}=\alpha]\]

\bookmarksetup{startatroot}

\chapter*{References}\label{references}
\addcontentsline{toc}{chapter}{References}

\markboth{References}{References}

\phantomsection\label{refs}
\begin{CSLReferences}{0}{1}
\end{CSLReferences}




\end{document}
